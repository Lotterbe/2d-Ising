
%\documentclass[10pt,a4paper]{article}
\documentclass[tikz]{standalone}
%\usepackage[utf8]{inputenc}

% TikZ und Bibliotheken
%\usepackage{tikz}
\usetikzlibrary{positioning,shapes.geometric}
\usetikzlibrary{backgrounds}
\usetikzlibrary{intersections}


\begin{document}

\tikzset{
  grec/.style={
  rectangle,
  fill=gray!20,
  draw=gray ,
  rounded corners
  },
  brec/.style={
  rectangle,
  fill=blue!20,
  draw=blue,
  rounded corners
  },
  orec/.style={
  rectangle,
  fill=orange!20,
  draw=orange ,
  rounded corners 
  },
  mrec/.style={
  rectangle,
  fill=magenta!20,
  draw=magenta,
  rounded corners
  },
   crec/.style={
  rectangle,
  fill=cyan!10,
  draw=cyan,
  rounded corners
  },
  reclayer/.style args={#1}{
   text = blue,
   dotted,
  rectangle,
  fill=black!5,
  draw=black ,
  label = {[black] #1}
  }
}

%\tikzset{
%io/.style={
%trapezium,
%trapezium left angle=70,
%trapezium right angle=110,
%fill=magenta!20,
% draw=magenta},
% op/.style={
%rectangle,
%fill=orange!20,
% draw=orange},
% co/.style={
%diamond,
%aspect=2,
%inner sep=2pt,
%fill=red!20,
% draw=red},
% node distance = 5mm}

%\subsection*{Newton Maschine}

\begin{tikzpicture}[thick, scale=3]
%\begin{scope}[draw, on background layer={color=black}, name path=backlayer] 
%\draw[yellow, thick] (-.4,-.4) rectangle (.4,.4);
%\node[grec] (force) {Kraft F};
%\end{scope}
%\scoped[on background layer]

%\draw[yellow, thick, name path=firstrec, label=Ursache] (-.4,-.4)  rectangle  (.4,.4 )node {};
\node[reclayer = Ursache, minimum width=2cm,minimum height=2cm] {};
%\node[above=of firstrec.north]  {Ursache};

%\draw[yellow, thick] (-.4,-.4) rectangle (.4,.4);
%\begin{draw}[yellow, thick]
 %\node[draw]  { \node[grec] (force) {Kraft F}};
%\end{draw}


%\node[draw,ultra thick,align=left,fill=orange]   {text};\hskip2pt

\node[mrec] (force) {Kraft F};
% draw=yellow!80!black, fill=yellow!20
\node[reclayer=System, xshift=2.5cm, minimum width=2.1cm,minimum height=2cm] {};

\node[grec, right=of force] (mass) {Masse m};

\node[reclayer=Wirkung, xshift=5.7cm, minimum width=3.4cm,minimum height=2cm] {};

\node[crec, right=of mass] (acc) {Beschleunigung a};

\path[->] (force) edge (mass)
 (mass) edge (acc);
\end{tikzpicture}

%\newpage

%\begin{tikzpicture}[thick, scale=2]
%\node[io] (vin) {Startwert $v_0$};
%\node[io] (sin) {Startwert $s_0$};


%\end{tikzpicture}




\end{document}