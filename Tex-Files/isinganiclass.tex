%\documentclass[10pt,a4paper]{article}
\documentclass[tikz]{standalone}
%\usepackage[utf8]{inputenc}

% TikZ und Bibliotheken
%\usepackage{tikz}
\usetikzlibrary{positioning,shapes.geometric}
\usetikzlibrary{calc}
\usetikzlibrary{backgrounds}
\usetikzlibrary{intersections}



\begin{document}

\tikzset{
  grec/.style={
  rectangle,
  fill=green!10,
  draw=green!50!black ,
  rounded corners
  },
  brec/.style={
  rectangle,
  fill=blue!20,
  draw=blue,
  rounded corners
  },
  orec/.style={
  rectangle,
  fill=orange!20,
  draw=orange ,
  rounded corners 
  },
  mrec/.style={
  rectangle,
  fill=magenta!20,
  draw=magenta,
  rounded corners
  },
   crec/.style={
  rectangle,
  fill=cyan!10,
  draw=cyan,
  rounded corners
  },
  reclayer/.style args={#1}{
   text = blue,
   dotted,
  rectangle,
  fill=black!5,
  draw=black ,
  label = {[black] #1}
  },
  io/.style={
trapezium,
trapezium left angle=70,
trapezium right angle=110,
fill=magenta!20,
 draw=magenta},
  op/.style={
rectangle,
fill=orange!20,
 draw=orange},
 co/.style={
diamond,
aspect=2,
inner sep=2pt,
fill=red!20,
 draw=red}
}

%\tikzset{
%io/.style={
%trapezium,
%trapezium left angle=70,
%trapezium right angle=110,
%fill=magenta!20,
% draw=magenta},
% op/.style={
%rectangle,
%fill=orange!20,
% draw=orange},
% co/.style={
%diamond,
%aspect=2,
%inner sep=2pt,
%fill=red!20,
% draw=red},
% node distance = 5mm}

%\subsection*{Newton Maschine}



\begin{tikzpicture}[thick, scale=3]

\node[reclayer = class, minimum width=4.8cm,minimum height=5.5cm]  (recone) {};

\node[crec, below=of recone.north, yshift=+0.75cm, xshift=-0.25cm, align=center, font=\huge] (force) {Ising\_Ani};

%\node[grec, below=of force, xshift=+0.5cm, yshift=0.9cm] (methodone) {measure\_observables()};



\node[grec, below=of force,  xshift=+0.2cm, yshift=0.8cm, align=left, font=\small] (methodone) {$\texttt{start\_animation():}$\\

\begin{tikzpicture}

\node[grec, yshift=0.7cm, align=left, font=\tiny] (update) {$\texttt{animation.FuncAnimation();}$} ;
\end{tikzpicture}

};

\path[->, draw] (force.west) -- ++ (-0.05, 0) |- (methodone);

%\node[grec, below=of methodtwo, yshift=0.7cm, align=right, font=\tiny] (update) { Choose random $s_{ij}$\\$\texttt{if } (e^{-\beta dE} \geq r):$ \\ $s_{ij} =(-1)\cdot s_{ij}$};


% (force.west) -- ++(-0.1, 0) |- edge (methodone.west);

\node[grec, below=of methodone,  xshift=-0.29cm, yshift=0.7cm, align=left, font=\small] (methodtwo) {


\begin{tikzpicture}

\node[grec, yshift=0.7cm, align=left, font=\tiny] (update) {$\texttt{update\_plot(num);}$} ;
\end{tikzpicture}\\

\begin{tikzpicture}

\node[grec, yshift=0.7cm, align=left, font=\tiny] (update) {$\texttt{update\_slider(val);}$} ;
\end{tikzpicture}

};
\path[->, draw] ($(methodtwo.south east) + (0.,0.08)$)  -- ++ (0.12,0) |- node[right, yshift=-0.3cm, align=left, font=\tiny] {every\\ frame} ($(methodtwo.east)+(0.0, 0.08)$);

\node[grec, below=of methodtwo,  xshift=+0.12cm, yshift=0.7cm, align=left, font=\small] (methodthree) {$\texttt{save(filename);}$%\\


%\begin{tikzpicture}
%
%\node[grec, yshift=0.7cm, align=left, font=\tiny] (update) {$\texttt{savez\_compressed();}$} ;
%\end{tikzpicture}

};

\path[->, draw] (force.west) -- ++ (-0.05, 0) |- (methodtwo);

\node[mrec, below=of methodthree, xshift=+0cm, yshift=0.7cm, align=left, font=\footnotesize] (result) {create $\texttt{.mp4 file}$ };

\path[->, draw] (force.west) -- ++ (-0.05, 0) |- (methodthree);

\path[->, draw] (methodone.south) -- ++(0,-0.1) ;%-- (methodtwo);
%\path[->, draw] (methodtwo) edge (methodthree);
\path[->, draw] (methodthree) edge (result);
  %{$\texttt{observables,}$ \\ $\texttt{configs}$}
  %{$\texttt{Neues Wertepaar}$ \\ $(t_{k+1}, s_{k+1})$}; 
\end{tikzpicture}


\end{document}